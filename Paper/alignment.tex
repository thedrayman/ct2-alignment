

\documentclass[conference]{IEEEtran}

\ifCLASSINFOpdf
   \usepackage[pdftex]{graphicx}
  \graphicspath{{../pdf/}{../jpeg/}}
\else
\fi

\usepackage{color,listings}
\usepackage{cite}

\lstset{language=[Sharp]C, % Grundsprache ist C und Dialekt ist Sharp (C#)
	captionpos=b, % Beschriftung ist unterhalb
	frame=lines, % Oberhalb und unterhalb des Listings ist eine Linie
	basicstyle=\ttfamily, % Schriftart
	keywordstyle=\color{blue}, % Farbe für die Keywords wie public, void, object u.s.w.
	commentstyle=\color{green}, % Farbe der Kommentare
	stringstyle=\color{red}, % Farbe der Zeichenketten
	numbers=left, % Zeilennummern links vom Code
	numberstyle=\tiny, % kleine Zeilennummern
	numbersep=5pt,
	breaklines=true, % Wordwrap a.k.a. Zeilenumbruch aktiviert
	showstringspaces=false,
	% emph legt Farben für bestimmte Wörter manuell fest
	emph={double,bool,int,unsigned,char,true,false,void},
	emphstyle=\color{blue},
	emph={Assert,Test},
	emphstyle=\color{red},
	emph={[2]\using,\#define,\#ifdef,\#endif}, emphstyle={[2]\color{blue}}
}

\usepackage{amsmath}
\usepackage{url}

\usepackage[latin1, utf8]{inputenc}

\hyphenation{op-tical net-works semi-conduc-tor}


\begin{document}

\title{Location Prediction with Alignment Algorithm on Google Location Data}

\author{\IEEEauthorblockN{Olga Groh\IEEEauthorrefmark{1},
Johann G\"otz\IEEEauthorrefmark{2},
Fabian Fr\"olich\IEEEauthorrefmark{3}}
\IEEEauthorblockA{Faculty of Electrical Engineering and Computer Science\\
University of Kassel,\\
Kassel, Germany\\
Email: {\{\IEEEauthorrefmark{1}o\_groh, \IEEEauthorrefmark{2}uk017305, \IEEEauthorrefmark{3}f.fr\"olich\}@student.uni-kassel.de }}}
	

\maketitle

\begin{abstract}

\end{abstract}

\IEEEpeerreviewmaketitle

\section{Introduction}\label{section:introduction}

%Voraussage der Location von alten Menschen und rechtzeitiges Handeln, wenn sie sich nicht an der entsprechenden Position befinden (Verhalten)

%Locations werden oft über bestimmte Applikationen automatisch von Smartphones aufgezeichnet

%Anbieten von personenbezogener Werbung an den vorhergesagten Standort

%Smart-Home Bereich (Temperatur, Energiemanagement, Komfort)


\section{Related Work}\label{section:relatedWork}
\cite{craig2017region}
%Alignment allgemein

%Location prediction allgemein

%Alignment eignet besonders gut für Locations

\section{Conception}\label{section:conception}

%Alignment wird normalerweise in der Biologie zum Vergleich von DNA-Sequenzen genutzt

%Wir nutzen Google Location Data, da Google es anbietet und fast jeder heutzutage ein Smartphone besitzt

%Alignment lässt sich auf Location data anwenden

%Idee der Implementierung



\section{Implementation}\label{section:implementation}

%Wie sehen die Daten aus?

%CSV Aufbau

%Import der CSV

\section{Evaluation}\label{section:evaluation}

%Wie schnell ist der Algorithmus auf großen Daten?

%Länge der Vergleichssequenz

%Grafiken/Bilder für vorhergesagte Locations (und alte Locations)


\section{Conclusion}\label{section:conclusion}

%Erneute Zusammenfassung der Ergebnisse

%Was ist dabei herausgekommen?

%Was könnte man noch machen?

%\newpage
\bibliographystyle{IEEEtran}
\bibliography{alignment}


\end{document}


