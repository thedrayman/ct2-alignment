

\documentclass[conference]{IEEEtran}

\ifCLASSINFOpdf
   \usepackage[pdftex]{graphicx}
  \graphicspath{{../pdf/}{../jpeg/}}
\else
\fi

\usepackage{color,listings}
\usepackage{cite}

\lstset{language=[Sharp]C, % Grundsprache ist C und Dialekt ist Sharp (C#)
	captionpos=b, % Beschriftung ist unterhalb
	frame=lines, % Oberhalb und unterhalb des Listings ist eine Linie
	basicstyle=\ttfamily, % Schriftart
	keywordstyle=\color{blue}, % Farbe für die Keywords wie public, void, object u.s.w.
	commentstyle=\color{green}, % Farbe der Kommentare
	stringstyle=\color{red}, % Farbe der Zeichenketten
	numbers=left, % Zeilennummern links vom Code
	numberstyle=\tiny, % kleine Zeilennummern
	numbersep=5pt,
	breaklines=true, % Wordwrap a.k.a. Zeilenumbruch aktiviert
	showstringspaces=false,
	% emph legt Farben für bestimmte Wörter manuell fest
	emph={double,bool,int,unsigned,char,true,false,void},
	emphstyle=\color{blue},
	emph={Assert,Test},
	emphstyle=\color{red},
	emph={[2]\using,\#define,\#ifdef,\#endif}, emphstyle={[2]\color{blue}}
}

\usepackage{amsmath}
\usepackage{url}

\usepackage{ngerman}
\usepackage[latin1, utf8]{inputenc}

\hyphenation{op-tical net-works semi-conduc-tor}


\begin{document}

\title{Location Prediction with Alignment Algorithm on Google Location Data}

\author{\IEEEauthorblockN{Olga Groh\IEEEauthorrefmark{1},
Johann G\"otz\IEEEauthorrefmark{2},
Fabian Fr\"olich\IEEEauthorrefmark{3}}
\IEEEauthorblockA{Faculty of Electrical Engineering and Computer Science\\
University of Kassel,\\
Kassel, Germany\\
Email: {\{\IEEEauthorrefmark{1}o\_groh, \IEEEauthorrefmark{2}uk017305, \IEEEauthorrefmark{3}f.fr\"olich\}@student.uni-kassel.de }}}
	

\maketitle

\begin{abstract}

\end{abstract}

\IEEEpeerreviewmaketitle

\section{Introduction}\label{section:introduction}
Das Wissen um die Position einer Person spielt in der heutigen, digitalen Welt eine wichtige Rolle. In allen denkbaren Bereichen kann eine genaue Position hilfreich sein oder gar Leben retten. Einen Schritt weiter geht die Positionsvorhersage, bei der der zuk�nfitige Aufenhaltsort einer Person bestimmt. Damit ergibt sich nicht nur das Wissen, wo die Person ist, sondern auch wo sie zu hoher Wahrscheinlichkeit als n�chstes sein wird. Um die Bedeutung dieser Technologie ein wenig zu verdeutlichen, seinen an dieser Stelle ein paar Szenarien beschrieben.

Ein lebensrettender Einsatz ist die Anwendung der Positionsbestimmung bzw. -vorhersage bei �lteren Menschen. Ausgesattet mit den richtigen Ger�ten, ist es heute m�glich den Gesundheitszustand einer Person zu erfassen und Aussagen �ber diese zu treffen. Abh�ngig von diesem Zustand kann dann bei einem Sturz oder einer Verschlechterung des Zustand Hilfe geholt werden. Dabei ist es wom�glich sehr wichtig, wo sich die Person befindet. �ber die Positionsbestimmtung ist es m�glich die Person zu finden - sofern diese Bestimmung aktuell ist. �ber die Positionsvorhersage ist es zus�tzlich m�glich Aussagen �ber den Aufenthaltsort zu machen, auch wenn das Ger�t der Positionsbestimmung vergessen wurde oder versagt.

Auch in anderen Gefahren- oder Rettungsszenarien ist die genaue Position hilfreich. Man stelle sich einen Unfall auf der Autobahn vor, dabei ist es f�r die Rettungskr�fte von sehr gro�en Vorteil zu Wissen auf welcher der Stra�e sich der Unfall ereignet hat. Hierbei ist die exakte Bestimmtung der Position wichtiger als die Vorhersage. Allerdings gibt es auch Situationen da ist eine Vorhersage f�r die Rettungskr�fte von Bedeutung. Als Beispiele seinen hier die Rettung von Wanderern bei schlechtem Wetter oder von Personen nach einer Naturkatastrophe genannt.

Weitere Situationen f�r die Wichtigkeit der Vorhersage von Positionen von Personen sind in der Verbrechensbek�mpfung und -aufkl�rung denkbar. F�r Unternehmen ist die Vorhersage ebenso ein wichtiges Instrument, lassen sich doch damit Werbeanzeigen personenbezogen anzeigen und verbreiten. Aber auch f�r eine Privatperson kann die Vorhersage der eingenen Position hilfreich sein, bedenkt man den steigenden Einsatz von Samrt-Home-Technik. Mit Vorhersagen �ber die Position des Eingent�mers, kann das heimische System wissen, wann die Person zu Hause eintrifft und somit die Heizung rechtzeitig einschalten.

All die genannten Szenarien sind abh�ngig von Positionsbestimmung und -vorhersage. Dabei ist es heutzutage kaum noch ein Problem die dazu ben�tigeten Daten zu erhalten. In den letzten zehn Jahren hat sich das Smartphone zu einem n�tzlichen und allzeits mitgef�hren Gef�hrten gemausert. Die M�chtigkeit des Begleiters wird trotzdem von den meisten untersch�tzt. Mit den erfassten, und in den meisten F�llen gespeichterten Daten, lassen sich Aussagen �ber den Besitzer treffen.Es l�sst sich gar ein ziemlich genaues pers�nliches Profil erstellen. Mit den erfassten Positionsdaten l�sst sich, mit Hilfe der richtigen Algorithmen und Methoden, eine Positionsvorhersage erstellen. 

In dieser Arbeit wird versucht, eine Positionsvorhersage mit dem Ansatz des Alignments zu erstellen. Die dabei genutzten, aufgezeichneten Positionsdaten wurden vorab durch verschiedene Smartphones und Personen erfasst. Mit Hilfe der Daten und einem Algorithmus des Alignments werden wir versuchen die n�chste Position vorherzusagen.
%Voraussage der Location von alten Menschen und rechtzeitiges Handeln, wenn sie sich nicht an der entsprechenden Position befinden (Verhalten)

%Locations werden oft über bestimmte Applikationen automatisch von Smartphones aufgezeichnet

%Anbieten von personenbezogener Werbung an den vorhergesagten Standort

%Smart-Home Bereich (Temperatur, Energiemanagement, Komfort)


\section{Related Work}\label{section:relatedWork}
\cite{craig2017region}
%Alignment allgemein

%Location prediction allgemein

%Alignment eignet besonders gut für Locations

\section{Conception}\label{section:conception}

%Alignment wird normalerweise in der Biologie zum Vergleich von DNA-Sequenzen genutzt

%Wir nutzen Google Location Data, da Google es anbietet und fast jeder heutzutage ein Smartphone besitzt

%Alignment lässt sich auf Location data anwenden

%Idee der Implementierung



\section{Implementation}\label{section:implementation}

%Wie sehen die Daten aus?

%CSV Aufbau

%Import der CSV

\section{Evaluation}\label{section:evaluation}

%Wie schnell ist der Algorithmus auf großen Daten?

%Länge der Vergleichssequenz

%Grafiken/Bilder für vorhergesagte Locations (und alte Locations)


\section{Conclusion}\label{section:conclusion}

%Erneute Zusammenfassung der Ergebnisse

%Was ist dabei herausgekommen?

%Was könnte man noch machen?

%\newpage
\bibliographystyle{IEEEtran}
\bibliography{alignment}


\end{document}


